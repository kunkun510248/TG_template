\documentclass[11pt,a4paper]{TGhomework}

\title{\textbf{数学分析第十周作业}}
\author{will}
\date{\today}

\begin{document}
\maketitle
\setlength{\parindent}{0pt}

\begin{problem}
习题12.5-1
\end{problem}

\begin{solution}
    记$A_k=A \cap[-k, k]^n$, 则由$\partial A_k$为零测集知:

    $\bigcup\limits_{k=1}^{+\infty} \partial A_k$为零测集.

    于是$\partial A \subset \bigcup\limits_{k=1}^{+\infty} \partial A_k$为零测集.

\end{solution}

\begin{problem}
习题12.5-2
\end{problem}

\begin{proof}
    $f$有界, 不妨设$|f| \geqslant M$

    $\int_{B_\varepsilon(p)}|f| \leq M \cdot V(B_\varepsilon)$ 有界.

    又 $f$ 在 $A-B_\varepsilon(p)$ 上可积, 则在$A$上可积, 且
    $$
        \int_A f=\int_{A - B-\varepsilon(p)} f+\int_{B_\varepsilon(p)} f
    $$
    而$\left|\int_{B_\varepsilon(p)} f\right| \leq M \cdot V\left(B_{\varepsilon}\right)$, 于是:
    $$
        \int_A f=\lim _{\varepsilon \to 0^{+}} \int_{A-B_\varepsilon(p)} f
    $$
从而得到原题结论.
\end{proof}

\begin{note}
    即得易见平凡, 仿照上例显然。留作习题答案略, 读者自证不难. 反之亦然同理, 推论自然成立。略去过程Q.E.D., 由上可知证毕.
\end{note}

\end{document}